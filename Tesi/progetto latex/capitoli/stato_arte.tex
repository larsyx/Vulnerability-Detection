\chapter{Background e Stato dell'arte} \label{1cap:background}
% [titolo ridotto se non ci dovesse stare] {titolo completo}
%

\markboth{Stato dell'arte}{}

\section{Vulnerabilità software}
Le vulnerabilità sono errori o componenti software dove le misure di sicurezza sono assenti o compromesse, questi spesso rendono l’intero sistema vulnerabile e possono essere sfruttati da malintenzionati per compiere possibili attacchi informatici \citep{Vulnerability}. Solitamente non sono le vulnerabilità in sé ad essere il problema e a danneggiare un sistema informatico, ma esse possono essere sfruttate per diffondere infezioni o possono essere il mezzo attraverso cui un malintenzionato può manipolare il sistema. Una vulnerabilità software diventa pericolosa quando questa viene scoperta infatti l’attaccante può sfruttare una vulnerabilità in modo tale da ottenere privilegi maggiori di quelli a lui concessi cosicché accedere a file o risorse a cui non potrebbe accedere, le vulnerabilità quindi costituiscono un’arma molto potente per gli hacker in quanto permettono di aggirare i sistemi di protezione rendendoli del tutto inutili e a consentire lo svolgimento, sul sistema attaccato, di operazioni non autorizzate.
Possiamo classificare le vulnerabilità informatiche in tre categorie:
\begin{itemize}
    \item \textbf{Vulnerabilità software \citep{Vulnerability}:} vengono definite anche bug software e sono malfunzionamenti legati a difetti di progettazione, codifica, installazione e configurazione del software. Tipicamente vengono introdotti tramite difetti nel codice o a causa di controlli non adeguati dei dati ricevuti in input dall'applicazione. Questa categoria di vulnerabilità è la più diffusa;
    \item \textbf{Vulnerabilità dei protocolli \citep{Vulnerability}:} si presentano quando protocolli di comunicazione hanno lacune di sicurezza nel sistema di comunicazione, un esempio è la comunicazione in chiaro o non crittografata che permette a possibili malintenzionati di intercettare le informazioni scambiate;
    \item \textbf{Vulnerabilità hardware \citep{Hardware_Vulnerability}:} quando in apparati hardware qualsiasi elemento può causare un pericolo al corretto funzionamento di una macchina tecnologica compromettendone la sicurezza, possiamo fare rifermento all'introduzione di umidità, polvere o qualsiasi cosa all'interno della macchina. \newline
\end{itemize}

Le vulnerabilità software sono tra le più diffuse, e sono anche il modo principale attraverso cui gli hacker compiono un attacco informatico, solitamente questo tipo di vulnerabilità viene introdotto durante le fasi del ciclo di vita del software in particolare durante le fasi di progettazione, programmazione, testing e installazione.\newline
Di notevole importanza sono le vulnerabilità zero-day \citep{Vulnerability_zero_day} queste sono vulnerabilità software scoperte dagli aggressori prima che il produttore del software ne sia venuto a conoscenza, ciò rende il software vulnerabile ad attacchi informatici finché non viene rilasciata una patch che corregge tale vulnerabilità. Un’ esempio molto discusso di vulnerabilità zero-day è stata scoperta a luglio del 2020 da parte di Opatch \citep{0patch} \citep{Forbes} sul software di videoconferenze \textit{Zoom}, questo software  rendeva vulnerabili a possibili attacchi informatici tutti i sistemi operativi Windows 7 e tutte le versioni precedenti. La vulnerabilità riscontrata fu classificata come Remote Code Execution (RCE) questa permetteva l’esecuzione di codice remoto sulla macchina attaccata e durante l'attacco l’utente non veniva avvisato, fortunatamente la vulnerabilità venne tempestivamente risolta da parte di Zoom dopo solo un giorno dalla sua segnalazione. \newline
Uno dei progetti open-source più importanti per la sicurezza web è l'OWASP \citep{OWASP} (Open Web Application Security Project) questo ha come scopo quello di fornire agli sviluppatori articoli, documenti e linee guida per l’implementazione e la realizzazione di sistemi software sicuri. Tra le iniziative più importanti intraprese dall’OWASP è la top 10 web application security risk, nella quale elenca le vulnerabilità software più comuni e rischiose che solitamente vengono introdotte nei sistemi software, inoltre per ognuna di esse fornisce una descrizione dei possibili rischi e di come evitare queste vulnerabilità.\newline
Esistono diversi tipi di vulnerabilità software i più comuni sono:
 
\begin{itemize}
    \item \textit{Buffer overflow \citep{Buffer_Overflow}:} avviene quando per errore o malizia vengono inviati dati in input ad un buffer più grandi della capienza del buffer stesso, i dati eccessivi sovrascrivono le variabili interne o lo stack della memoria, questo errore compromette il programma, il quale può avere un comportamento imprevisto o bloccarsi del tutto, se vengono immessi dati malevoli essi possono prendere il controllo del programma o peggio il controllo del intero computer. Non tutti i linguaggi di programmazione possono riscontrare questa vulnerabilità; 
    
    \item \textit{Injection \citep{OWASP_top_10}:} questo difetto permette agli aggressori l’iniezione di codice tramite chiamate di sistema, queste solitamente vengono eseguite utilizzando programmi esterni tramite comandi della shell. Le iniezioni al database in particolare le iniezioni SQL injection solo le comuni e pericolose; 
    
    \item \textit{Broken Authentication \citep{OWASP_top_10}:} questo tipo di vulnerabilità si verifica quando un malintenzionato riesce ad utilizzare modi diversi per entrare nel account di qualcun altro. Questo tipo di vulnerabilità colpisce principalmente le applicazioni web, le cause principali sono la mancanza di verifiche o timeout in una sessione. Un altro modo è quando l’hacker riesce ad accedere al database delle password e sfruttando altre vulnerabilità riesce a convertire la codifica di queste permettendogli di visualizzare le password di tutti gli utenti; 
    
    \item \textit{Broken Access Control:} questa vulnerabilità porta alla mancanza di controllo su chi ha accesso per leggere e modificare i dati. Nella maggior parte delle volte questa vulnerabilità porta ad accesso, utilizzo, modifica o divulgazione di dati non autorizzati; 
    
    \item \textit{Criptographic failure \citep{OWASP_top_10}:} questo tipo di vulnerabilità espone dati sensibili come password, informazioni personali, numeri di carte di credito etc.., esso solitamente è derivato da usi scorretti della crittografia, da algoritmi crittografici deboli o da trasmissioni di dati in chiaro su protocolli di comunicazione come FTP o HTTP;
    
    \item \textit{Software and data Integrity failures \citep{OWASP_top_10}:} è riferito a codice che non protegge da violazioni di integrità. Solitamente questo tipo di vulnerabilità viene introdotta quando un’ applicazione si basa su plug-in o librerie che provengono da fonti non attendibili, un altro tipico esempio è quando applicazioni che offrono servizio di aggiornamento automatico, alcuni di essi non effettuano opportuni controlli sul integrità dei dati sui nuovi aggiornamenti, applicandoli ad applicazioni precedentemente attendibili. Questo codice può introdurre codice dannoso che potrebbe compromettere l’integrità del sistema.
\end{itemize}

Gli attacchi informatici più comuni sono definiti crimini di dati \citep{saini2012cyber}, dove gli attaccanti hanno come obiettivo quello di rubare, manipolare e intercettare dati per trarne profitto, un'altra tipologia molto diffusa di crimine informatico è la disseminazione di virus \citep{saini2012cyber}, i virus sono software dannosi che hanno come obiettivo quello di distruggere il sistema della  vittima. Tutti questi crimini  hanno un’ impatto  sulle aziende  e sugli utenti molto forte soprattutto in termini economici ma anche di sicurezza e privacy \citep{saini2012cyber}.

Nonostante tutti gli sforzi fatti per la prevenzione e per l’individuazione delle vulnerabilità software, spesso queste tecniche sono molto costose, il che rende la prevenzione e la risoluzione di vulnerabilità metodi poco efficienti, per questo motivo nell'arco degli anni diversi studi si sono concentrati sull’individuazione delle vulnerabilità attraverso tecniche di predizione, in particolare attraverso l’uso di machine learning.



\section{Modelli di apprendimento per predizione di vulnerabilità}

Come abbiamo visto le vulnerabilità software sono una problematica molto importante da tenere in considerazione durante lo sviluppo e ciclo di vita di sistema software, sfortunatamente a causa delle risorse limitate non è possibile condurre un controllo dettagliato in fase di sviluppo o di testing per trovare e risolvere vulnerabilità software, per questo motivo gli sviluppatori negli ultimi anni si sono concentrati sulla creazione di strumenti in grado di automatizzare la predizione di vulnerabilità \citep{shamal2017study}, tramite l’intelligenza artificiale, in particolare nell'ambito del machine learning sono stati compiuti molti studi per cercare di risolvere questo problema, in questo campo un’attenzione particolare è stata posta sui modelli di predizione di vulnerabilità software (VPM). \newline
Nel machine learning la scelta delle ''software feature'' gioca un ruolo molto importante per il corretto funzionamento e per migliorare gli indici di prestazione, per cui uno dei primi obiettivi dei VPM consiste nell’aumentare l’accuratezza della previsione limitando i costi ed estraendo attributi software appropriati \citep{yang2016improving}. \newline 
Una prima categoria usata in vari algoritmi sono i predittori basati sulle metriche software, queste rappresentano tramite valori numerici caratteristiche di un software, questa tipologia di metriche considerano che al crescere della complessità e della grandezza di un sistema software esso diventa potenzialmente sempre più esposto a vulnerabilità software, per questo motivo solitamente vengono prese in considerazione metriche che considerano la complessità del codice tra i principali abbiamo: numero di funzioni, linee di codice (LOC), numero di chiamate a metodi o funzioni esterne, numero di chiamate a metodi o funzioni interni ed esterni, numero di nidificazione e numero di file (Fan-in e Fan-out) \citep{walden2014predicting}. \newline
Uno dei primi studi è stato effettuato da Zimmerman et al. \citep{zimmermann2010searching} in questo studio sono stati effettuati una serie di test empirici su Windows Vista in cui hanno provato l'efficienza di varie metriche software classiche come metriche di complessità, code churn e misure di dipendenza, hanno riscontrato un'alta precisione ma un basso recall. Similmente Shin et al. \citep{shin2010evaluating} crearono una serie di modelli divisi in tre categorie: complessità, code churn e sulle attività degli sviluppatori, gli studi furono effettuati su Mozilla Firefox web browser e the Red Hat Enterprise Linux kernel, in totale hanno ottenuto e testato 28 metriche e 24 di queste sono riuscite a discriminare vulnerabilità software. Neuahus et al. \citep{neuhaus2007predicting} nel contesto di Morzilla Firefox crearono un nuovo strumento per la predizione di vulnerabilità chiamato ''Vulture'', il suo funzionamento prevede l'associazione di vulnerabilità passate a componenti software tramite un database di vulnerabilità, il predittore risultante prevede vulnerabilità dei nuovi componenti in base agli import o alle funzioni chiamate, Vulture riscontrò un'alta precisione 70\% ma un basso recall 45\%. Nguyen et al. \citep{nguyen2010predicting} proposero un nuova metrica basata sui grafi delle dipendenze delle componenti  di un sistema software (CDG), questi grafi sono basati sulle relazioni tra gli elementi software come classi, funzioni o variabili, ciò rende i grafi estraibili tramite un'analisi del codice sorgente o dai dettagli delle specifiche di design, per questo motivo questa metrica è possibile usarla sia in fase di sviluppo che in fase di design. Questo modello fu sperimentato su JavaScript Engine of Firefox ed ottennero risultati migliori confrontandoli con i modelli basati su metriche di complessità, ottenendo un'alta accuratezza di circa 84\% e un alto recall di circa 60\%. \newline
Scandariato et al. \citep{scandariato2014predicting} furono i primi a proporre un modello di predizione delle vulnerabilità basato sul text mining, usarono la tecnica del ''bags of words'', dove hanno estratto un insieme di termini più frequenti nel codice sorgente Java e posti in un dataset vennero usati come features per la predizione del algoritmo usato, lo studio fu condotto su 20 applicazioni android, ed ottennero una precisione e un recall di circa l'80\%, 
Walden et al. tramite uno studio \citep{walden2014predicting} cercarono di fare confronto tra le classiche metriche software e il text mining. Il test fu effettuato nelle stesse condizioni per entrambe le metriche, il dataset era composto da 223 vulnerabilità riscontrate in applicazioni web, fu usato il classificatore Random Forest e furono testati su tre progetti. Il text mining riscontrò una maggiore precisione e un maggiore recall rispetto alle metriche software, tuttavia riscontrarono pessimi risultati durante la predizione su un progetto usando un predittore sviluppato su un altro progetto, in aggiunta si riscontrò che in entrambi i modelli all'aumento della grandezza del prodotto si verificò una perdita di precisione.\newline
Come mostrato da Theisen e Williams \citep{theisen2020better} la combinazione di feature da più modelli non necessariamente aumenta l'accuratezza del predittore anzi in alcuni casi si ha un degrado di prestazioni quindi il migliore approccio per migliorare le prestazioni di un VPM consiste nel trovare il miglior classificatore e le migliori feature da dargli in input.

\section{Dataset di apprendimento per predizione di vulnerabilità }
Uno degli elementi più importanti di un machine learning è un dataset, infatti quest'ultimo è il uno dei punti cardini per l'implementazione e il corretto funzionamento di un machine learning, per questo motivo la scarsa qualità di un dataset può influire negativamente sulla qualità della  predizione prodotta.\newline
I dataset presenti attualmente possono essere classificati in base al loro target di prodotti \citep{massacci2010right}, la prima tipologia sono i dataset multi-vendor, questi hanno la caratteristica di includere un largo numero di vulnerabilità, riscontrabili in qualsiasi tipo di ambiente software, tra i più importanti ci sono:
\begin{itemize}
    \item \textbf{CVE} (Common Vulnerabilities Exposure) \citep{CVE}: questo dataset è pubblico ed è gestito da Mitre Corporation, è nato con lo scopo di identificare e catalogare in un database pubblico vulnerabilità software. Ogni vulnerabilità del dataset ha un ID, una piccola descrizione e dei riferimenti. La pubblicazione di nuove vulnerabilità può avvenire direttamente dal MITRE o attraverso i CNA (CVE Numbering Authority) che rappresentano i principali fornitori IT o enti di ricerca (Microsoft, Red Hat, Oracle, etc.);
    \item \textbf{NVD} (National Vulnerability Database) \citep{NVD}: è un progetto del dipartimento di sicurezza nazionale degli Stati Uniti ed è stato creato con lo scopo di aiutare persone e aziende nel processo di automatizzazione delle vulnerabilità software. I dati sono rappresentati tramite il Security Content Automation Protocol (SCAP), questo protocollo associa ad ogni vulnerabilità presente nel database un gran numero di dati e informazioni utili per la gestione automatizzata delle vulnerabilità, tra i più importanti troviamo il CVSS (Common Vulnerability Scoring System) che assegna un punteggio alla gravità della vulnerabilità esaminata, il CWE (Common Weakness Enumeration) che descrive il tipo di vulnerabilità e il CPE (Common Platform Enumeration) che descrive la classe di applicazioni e sistemi operativi dove la vulnerabilità in questione è riscontrabile. Le vulnerabilità vengono aggiunte al NVD a partire da CVE, infatti queste vengono analizzate e integrate con il database;
    \item \textbf{OSVDB} (Open Source Vulnerability Database): è un database indipendente e open-source nato con lo scopo di descrivere in modo accurato le vulnerabilità software.
\end{itemize}

La seconda categoria sono i dataset vendor, questi sono creati appositamente per un applicazione o un sistema operativo, in questi troviamo tutte le vulnerabilità riscontrabili in quella tipologia di prodotti, tra i più importanti troviamo:
\begin{itemize}
    \item \textbf{MFSA} (Mozilla Foundation Security Advisories) \citep{MFSA}: sono tutti i report riscontrati nei prodotti Mozilla, ad ogni report vengono elencati tutte le vulnerabilità riscontrate e corrette, per ognuna di queste viene assegnato un valore di impatto e una descrizione;
    \item  \textbf{Bugzilla}: esistono due istanze di Bugzilla, RedHat Linux Bugzilla \citep{RedHat_Linux_Bugzilla} che comprende tutte le vulnerabilità del sistema operativo RedHat Linux, Mozilla Bugzilla \citep{Mozilla_Bugzilla} invece è un database che comprende tutte le vulnerabilità presenti in tutti i prodotti Mozilla. 
\end{itemize}

\newpage