\chapter{Conclusioni e sviluppi futuri} \label{1cap:conclusione}
% [titolo ridotto se non ci dovesse stare] {titolo completo}
%


In conclusione questo studio ha mostrato come e perché le vulnerabilità software coprono un ruolo di fondamentale importanza sul ciclo di vita del software e come queste possono influire notevolmente sulla vita degli utilizzatori, l'obiettivo principale di questo studio è stato di comprendere come le vulnerabilità possono essere individuate attraverso l'uso di VPM (modelli di predizione di vulnerabilità) analizzando come quest'ultimi lavorano e comprendendo il loro processo di implementazione. La parte più importante di questo lavoro è stata l'implementazione di un modello di predizione basato su machine learning che funzionasse sulla base di diverse metriche software, dalla costruzione del modello ho compreso come diverse metriche software possono lavorare insieme e quali di queste hanno un impatto maggiore, analizzando la loro correlazione con la presenza di vulnerabilità software. Dalla costruzione del modello ho potuto notare come i diversi classificatori lavorano e quali di questi hanno avuto un impatto migliore in termini di prestazioni per la predizione e il motivo legato questa differenza. \newline
Questo lavoro apre molte porte di sviluppo, essendo come punto di partenza per altri studi, uno dei lavori futuri più importanti potrebbe essere l'ottimizzazione del modello in termini di correttezza delle predizioni, questa ottimizzazione potrebbe essere sviluppata attraverso l'uso diversi algoritmi di ottimizzazione e di ricerca. Un importante lavoro potrebbe essere l'implementazione di un prodotto software sul quale installare il modello di predizione e che renda l'utilizzo del modello più semplice e veloce, e fruibile su più dispositivi. Questo lavoro per varie vicissitudini ha compreso ed analizzato un numero ridotto di metriche software anche se ne esistono un numero molto più grande ed alcune di queste raccolgono informazioni molto interessanti e che possono avere un impatto maggiore sulla corretta predizione del modello, per cui uno dei lavori più importanti è l'analisi di altre metriche software, con complessità diverse tra loro. Un importante sviluppo potrebbe essere la costruzione di un nuovo dataset, sulla base di quello usato da questo studio, che comprende diversi progetti con diverse complessità, integrando anche progetti software legati ad altri linguaggi di programmazione, in modo tale da creare un modello di predizione più preciso e complesso, il quale operi in modo corretto su più tipologie di prodotti software.